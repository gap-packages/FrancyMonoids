% generated by GAPDoc2LaTeX from XML source (Frank Luebeck)
\documentclass[a4paper,11pt]{report}

\usepackage{a4wide}
\sloppy
\pagestyle{myheadings}
\usepackage{amssymb}
\usepackage[utf8]{inputenc}
\usepackage{makeidx}
\makeindex
\usepackage{color}
\definecolor{FireBrick}{rgb}{0.5812,0.0074,0.0083}
\definecolor{RoyalBlue}{rgb}{0.0236,0.0894,0.6179}
\definecolor{RoyalGreen}{rgb}{0.0236,0.6179,0.0894}
\definecolor{RoyalRed}{rgb}{0.6179,0.0236,0.0894}
\definecolor{LightBlue}{rgb}{0.8544,0.9511,1.0000}
\definecolor{Black}{rgb}{0.0,0.0,0.0}

\definecolor{linkColor}{rgb}{0.0,0.0,0.554}
\definecolor{citeColor}{rgb}{0.0,0.0,0.554}
\definecolor{fileColor}{rgb}{0.0,0.0,0.554}
\definecolor{urlColor}{rgb}{0.0,0.0,0.554}
\definecolor{promptColor}{rgb}{0.0,0.0,0.589}
\definecolor{brkpromptColor}{rgb}{0.589,0.0,0.0}
\definecolor{gapinputColor}{rgb}{0.589,0.0,0.0}
\definecolor{gapoutputColor}{rgb}{0.0,0.0,0.0}

%%  for a long time these were red and blue by default,
%%  now black, but keep variables to overwrite
\definecolor{FuncColor}{rgb}{0.0,0.0,0.0}
%% strange name because of pdflatex bug:
\definecolor{Chapter }{rgb}{0.0,0.0,0.0}
\definecolor{DarkOlive}{rgb}{0.1047,0.2412,0.0064}


\usepackage{fancyvrb}

\usepackage{mathptmx,helvet}
\usepackage[T1]{fontenc}
\usepackage{textcomp}


\usepackage[
            pdftex=true,
            bookmarks=true,        
            a4paper=true,
            pdftitle={Written with GAPDoc},
            pdfcreator={LaTeX with hyperref package / GAPDoc},
            colorlinks=true,
            backref=page,
            breaklinks=true,
            linkcolor=linkColor,
            citecolor=citeColor,
            filecolor=fileColor,
            urlcolor=urlColor,
            pdfpagemode={UseNone}, 
           ]{hyperref}

\newcommand{\maintitlesize}{\fontsize{50}{55}\selectfont}

% write page numbers to a .pnr log file for online help
\newwrite\pagenrlog
\immediate\openout\pagenrlog =\jobname.pnr
\immediate\write\pagenrlog{PAGENRS := [}
\newcommand{\logpage}[1]{\protect\write\pagenrlog{#1, \thepage,}}
%% were never documented, give conflicts with some additional packages

\newcommand{\GAP}{\textsf{GAP}}

%% nicer description environments, allows long labels
\usepackage{enumitem}
\setdescription{style=nextline}

%% depth of toc
\setcounter{tocdepth}{1}





%% command for ColorPrompt style examples
\newcommand{\gapprompt}[1]{\color{promptColor}{\bfseries #1}}
\newcommand{\gapbrkprompt}[1]{\color{brkpromptColor}{\bfseries #1}}
\newcommand{\gapinput}[1]{\color{gapinputColor}{#1}}


\begin{document}

\logpage{[ 0, 0, 0 ]}
\begin{titlepage}
\mbox{}\vfill

\begin{center}{\maintitlesize \textbf{ FrancyMonoids \mbox{}}}\\
\vfill

\hypersetup{pdftitle= FrancyMonoids }
\markright{\scriptsize \mbox{}\hfill  FrancyMonoids  \hfill\mbox{}}
{\Huge \textbf{ FrancyMonoids/A package to display commutative monoid objects with francy \mbox{}}}\\
\vfill

{\Huge  0.1 \mbox{}}\\[1cm]
{ 4 June 2018 \mbox{}}\\[1cm]
\mbox{}\\[2cm]
{\Large \textbf{ Garcia-Sanchez Pedro A.\\
    \mbox{}}}\\
{\Large \textbf{ Herrera-Poyatos Andres\\
    \mbox{}}}\\
{\Large \textbf{ Martins Manuel\\
   \mbox{}}}\\
\hypersetup{pdfauthor= Garcia-Sanchez Pedro A.\\
    ;  Herrera-Poyatos Andres\\
    ;  Martins Manuel\\
   }
\end{center}\vfill

\mbox{}\\
{\mbox{}\\
\small \noindent \textbf{ Garcia-Sanchez Pedro A.\\
    }  Email: \href{mailto://pedro@ugr.es} {\texttt{pedro@ugr.es}}\\
  Homepage: \href{https://www.ugr.es/~pedro} {\texttt{https://www.ugr.es/\texttt{\symbol{126}}pedro}}\\
  Address: \begin{minipage}[t]{8cm}\noindent
 Departamento de {\a'A}lgebra, Facultad de Ciencias\\
 Universidad de Granada, 18071 Granada\\
 Spain\\
 \end{minipage}
}\\
{\mbox{}\\
\small \noindent \textbf{ Herrera-Poyatos Andres\\
    }  Email: \href{mailto://andreshp99@gmailcom} {\texttt{andreshp99@gmailcom}}\\
  Homepage: \href{https://github.com/andreshp} {\texttt{https://github.com/andreshp}}\\
  Address: \begin{minipage}[t]{8cm}\noindent
 Departamento de {\a'A}lgebra, Facultad de Ciencias\\
 Universidad de Granada, 18071 Granada\\
 Spain\\
 \end{minipage}
}\\
{\mbox{}\\
\small \noindent \textbf{ Martins Manuel\\
   }  Email: \href{mailto://manuelmachadomartins@gmail.com} {\texttt{manuelmachadomartins@gmail.com}}\\
  Homepage: \href{http://github.com/mcmartins} {\texttt{http://github.com/mcmartins}}}\\
\end{titlepage}

\newpage\setcounter{page}{2}
\newpage

\def\contentsname{Contents\logpage{[ 0, 0, 1 ]}}

\tableofcontents
\newpage

     
\chapter{\textcolor{Chapter }{FrancyMonoids automatic generated documentation}}\label{Chapter_FrancyMonoids_automatic_generated_documentation}
\logpage{[ 1, 0, 0 ]}
\hyperdef{L}{X860A68B479E58E03}{}
{
  
\section{\textcolor{Chapter }{FrancyMonoids automatic generated documentation of methods}}\label{Chapter_FrancyMonoids_automatic_generated_documentation_Section_FrancyMonoids_automatic_generated_documentation_of_methods}
\logpage{[ 1, 1, 0 ]}
\hyperdef{L}{X7A7F12647DBD2C49}{}
{
  

\subsection{\textcolor{Chapter }{DrawFactorizationGraph (for IsRectangularTable)}}
\logpage{[ 1, 1, 1 ]}\nobreak
\hyperdef{L}{X7BCF8A9C80766478}{}
{\noindent\textcolor{FuncColor}{$\triangleright$\enspace\texttt{DrawFactorizationGraph({\mdseries\slshape f})\index{DrawFactorizationGraph@\texttt{DrawFactorizationGraph}!for IsRectangularTable}
\label{DrawFactorizationGraph:for IsRectangularTable}
}\hfill{\scriptsize (operation)}}\\
\textbf{\indent Returns:\ }
a drawing 



 f is a set of factorizations Draws the graph of factorizations associated to
f: a complete graph {\nobreakspace}whose vertices are the elements of f. Edges
are labelled with distances between nodes they join. Kruskal algorithm is used
to draw in red a spannin tree with minimal distances. Thus the catenary degree
is reached in the edges of the tree. }

 

\subsection{\textcolor{Chapter }{DrawEliahouGraph (for IsRectangularTable)}}
\logpage{[ 1, 1, 2 ]}\nobreak
\hyperdef{L}{X7D6ED35C7DDC0441}{}
{\noindent\textcolor{FuncColor}{$\triangleright$\enspace\texttt{DrawEliahouGraph({\mdseries\slshape f})\index{DrawEliahouGraph@\texttt{DrawEliahouGraph}!for IsRectangularTable}
\label{DrawEliahouGraph:for IsRectangularTable}
}\hfill{\scriptsize (operation)}}\\
\textbf{\indent Returns:\ }
a drawing 



 f is a set of factorizations {\nobreakspace}Draws the Eliahou graph of
factorizations associated to f: a graph {\nobreakspace}whose vertices are the
elements of f, and there is an edge between two vertices if they have common
support. Edges are labelled with distances between nodes they join. }

 

\subsection{\textcolor{Chapter }{DrawRosalesGraph (for IsHomogeneousList,IsAffineSemigroup)}}
\logpage{[ 1, 1, 3 ]}\nobreak
\hyperdef{L}{X7BAFA5FD8587320E}{}
{\noindent\textcolor{FuncColor}{$\triangleright$\enspace\texttt{DrawRosalesGraph({\mdseries\slshape n, s})\index{DrawRosalesGraph@\texttt{DrawRosalesGraph}!for IsHomogeneousList,IsAffineSemigroup}
\label{DrawRosalesGraph:for IsHomogeneousList,IsAffineSemigroup}
}\hfill{\scriptsize (operation)}}\\
\textbf{\indent Returns:\ }
a drawing 



 s is either a numerical semigroup or an affine semigroup, and n is an element
of s Draws the graph associated to n in s. }

 

\subsection{\textcolor{Chapter }{DrawRosalesGraph (for IsInt,IsNumericalSemigroup)}}
\logpage{[ 1, 1, 4 ]}\nobreak
\hyperdef{L}{X81516F4E7F220EAA}{}
{\noindent\textcolor{FuncColor}{$\triangleright$\enspace\texttt{DrawRosalesGraph({\mdseries\slshape arg1, arg2})\index{DrawRosalesGraph@\texttt{DrawRosalesGraph}!for IsInt,IsNumericalSemigroup}
\label{DrawRosalesGraph:for IsInt,IsNumericalSemigroup}
}\hfill{\scriptsize (operation)}}\\


 

 }

 }

 
\section{\textcolor{Chapter }{FrancyMonoids automatic generated documentation of global functions}}\label{Chapter_FrancyMonoids_automatic_generated_documentation_Section_FrancyMonoids_automatic_generated_documentation_of_global_functions}
\logpage{[ 1, 2, 0 ]}
\hyperdef{L}{X8769AEF77920E73C}{}
{
  

\subsection{\textcolor{Chapter }{DrawOverSemigroupsNumericalSemigroup}}
\logpage{[ 1, 2, 1 ]}\nobreak
\hyperdef{L}{X8575E9C68739B408}{}
{\noindent\textcolor{FuncColor}{$\triangleright$\enspace\texttt{DrawOverSemigroupsNumericalSemigroup({\mdseries\slshape s})\index{DrawOverSemigroupsNumericalSemigroup@\texttt{Draw}\-\texttt{Over}\-\texttt{Semigroups}\-\texttt{Numerical}\-\texttt{Semigroup}}
\label{DrawOverSemigroupsNumericalSemigroup}
}\hfill{\scriptsize (function)}}\\
\textbf{\indent Returns:\ }
a drawing 



 Draws the Hasse diagram of {\nobreakspace} oversemigroupstree of the numerical
semigroup s. Irreducible numerical semigroups are highlighted. }

 

\subsection{\textcolor{Chapter }{DrawTreeOfSonsOfNumericalSemigroup}}
\logpage{[ 1, 2, 2 ]}\nobreak
\hyperdef{L}{X83B9B9F47F8D6C4C}{}
{\noindent\textcolor{FuncColor}{$\triangleright$\enspace\texttt{DrawTreeOfSonsOfNumericalSemigroup({\mdseries\slshape s, l, generators})\index{DrawTreeOfSonsOfNumericalSemigroup@\texttt{DrawTreeOfSonsOfNumericalSemigroup}}
\label{DrawTreeOfSonsOfNumericalSemigroup}
}\hfill{\scriptsize (function)}}\\
\textbf{\indent Returns:\ }
a drawing 



 Draws the tree of sons of numerical semigroups up to level l with
{\nobreakspace} respect to the minimal system of generators given by the
function generators. }

 

\subsection{\textcolor{Chapter }{DrawHasseDiagramOfNumericalSemigroup}}
\logpage{[ 1, 2, 3 ]}\nobreak
\hyperdef{L}{X7D16EFDD7F2CC7C9}{}
{\noindent\textcolor{FuncColor}{$\triangleright$\enspace\texttt{DrawHasseDiagramOfNumericalSemigroup({\mdseries\slshape s, A})\index{DrawHasseDiagramOfNumericalSemigroup@\texttt{Draw}\-\texttt{Hasse}\-\texttt{Diagram}\-\texttt{Of}\-\texttt{Numerical}\-\texttt{Semigroup}}
\label{DrawHasseDiagramOfNumericalSemigroup}
}\hfill{\scriptsize (function)}}\\
\textbf{\indent Returns:\ }
a drawing 



 plots a graph whose set of vertices is A, which is a finite set of integers,
and whose edges are provided by the order of the numerical semigroup s. }

 

\subsection{\textcolor{Chapter }{DrawTreeOfGluingsOfNumericalSemigroup}}
\logpage{[ 1, 2, 4 ]}\nobreak
\hyperdef{L}{X7DE89AFA803BBDDF}{}
{\noindent\textcolor{FuncColor}{$\triangleright$\enspace\texttt{DrawTreeOfGluingsOfNumericalSemigroup({\mdseries\slshape s[, expand]})\index{DrawTreeOfGluingsOfNumericalSemigroup@\texttt{Draw}\-\texttt{Tree}\-\texttt{Of}\-\texttt{Gluings}\-\texttt{Of}\-\texttt{Numerical}\-\texttt{Semigroup}}
\label{DrawTreeOfGluingsOfNumericalSemigroup}
}\hfill{\scriptsize (function)}}\\
\textbf{\indent Returns:\ }
a drawing 



 Returns a Francy canvas with the tree of gluings of the numerical semigroup s.
If the optional argument expand is provided, then the tree is drawn fully
expanded. }

 }

 }

 \def\indexname{Index\logpage{[ "Ind", 0, 0 ]}
\hyperdef{L}{X83A0356F839C696F}{}
}

\cleardoublepage
\phantomsection
\addcontentsline{toc}{chapter}{Index}


\printindex

\immediate\write\pagenrlog{["Ind", 0, 0], \arabic{page},}
\newpage
\immediate\write\pagenrlog{["End"], \arabic{page}];}
\immediate\closeout\pagenrlog
\end{document}
